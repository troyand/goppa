\documentclass[a4paper,12pt,oneside]{article}
\usepackage[left=30mm,right=15mm,top=20mm,bottom=20mm,includefoot]{geometry}
\usepackage[small,compact,center]{titlesec}
\renewcommand{\subsubsection}[1]{\addcontentsline{toc}{subsubsection}{\arabic{section}.\arabic{subsection}.\arabic{
    subsubsection}.~#1}\stepcounter{subsubsection}\textbf{\arabic{section}.\arabic{subsection}.\arabic{subsubsection}.~#1}}
\usepackage{setspace}
\onehalfspacing
\usepackage[utf8]{inputenc}
\usepackage[english,ukrainian]{babel}
\usepackage{verbatim}
\usepackage{indentfirst}
\usepackage{fullpage}
\usepackage{longtable}
\usepackage{amsmath}
\usepackage{cases}
\usepackage{graphicx}
\usepackage{cite}

\title{Курсова робота\\Надлишкове кодування на еліптичних кривих}
\author{Михайло Трояновський}

\begin{document}

\begin{titlepage}
\begin{center}
Національний Університет\\ «Києво-Могилянська Академія»\\
        Кафедра математики\\
        \vskip 3cm
        \includegraphics[width=100mm]{kma}\\
                               \vskip 1cm
{\Large
    Курсова робота на тему:\\
        «Надлишкове кодування на еліптичних кривих»
}
\vskip 1cm
\hskip 0.5\textwidth
\begin{minipage}{0.5\textwidth}
\textbf{Виконав:}\\
        Михайло Михайлович Трояновський, студент ІСПР–1\\
        \\
        \textbf{Науковий керівник:}\\
        Юрій Вікторович Боднарчук
        \end{minipage}
        \vfill
        Київ–2010
        \end{center}
        \end{titlepage}


\tableofcontents
\pagebreak

\section{Вступ}
Надлишкове кодування відіграє значну роль у боротьбі з помилками, які виникають при передачі даних через зашумлені телекомунікаційне середовище. 
Загалом для надійної передачі інформації через ненадійні канали застосовують дві стратегії: виявлення та виправлення помилок. 
У першому випадку до блоку даних додають контрольну інформацію, за допомогою якої одержувач має змогу зрозуміти, 
що під час зв'язку відбулося пошкодження відповідного блоку, тому необхідно зробити запит на повторну передачу. 
Найпростішим прикладом такої контрольної інформації є біт парності, який додається до блоку бінарних даних таким чином, аби загальна кількість одиниць у блоці була парною. 
Це дає можливість одержувачу виявити помилку в одному біті переданого блоку, проте якщо помилок було більше, вони можуть скасувати одна одну; 
втім подібний метод можна досить просто узагальнити для виявлення будь-якої непарної кількості помилок.

Проте існують застосування, у яких одного виявлення факту помилки при передачі інформації замало, оскільки повторна передача блоку даних може бути занадто витратною: 
так при зв'язку із об'єктами у космосі, що віддалені від Землі на велику відстань існує значна затримка, яка буде лише збільшуватися при ретрансмісії; 
при односторонній передачі даних (наприклад, у телемовленні) ретрансмісія взагалі технічно неможлива. 
У таких випадках необхідно намагатися виправляти помилки наперед, додаючи до блоку даних надлишкову інформацію, яка дозволятиме одержувачу відновити блок навіть за наявності помилок передачі. 
Простим прикладом даного підходу може бути таке бінарне кодування: кожен біт передається тричі підряд, одержувач аналізує триплети та декодує їх у той біт, якого кількісно більше у триплеті. 
Зрозуміло, що такий спосіб дозволяє виправити одну помилку в окремому триплеті, але ціна такої можливості досить велика: дані за такого підходу будуть передаватися втричі повільніше.

Природнім є бажання знайти способи надлишкового кодування, які з одного боку дозволятимуть виправляти значну кількість помилок, а з іншого боку не призводить до сповільнення передачі. 
Розвиненою в цьому аспекті є теорія лінійних кодів, де були встановлені нерівності, які дають можливість порівняти різні кодування між собою, 
а також обирати для практичного використання коду із граничними характеристиками. 
Зокрема, важливою є границя Сінглтона, яка стверджує, що для лінійного $(n, k, d)$-коду справжується нерівність: $d \le n-k+1$. 
Код який досягає відповідної рівності, називають кодом із максимально досяжною мінімальною відстанню, нетривіальним прикладом такого випадку слугують коди Ріда-Соломона.

\section{Постановка задачі}
Мета даної курсової роботи -- описати наявні способи надлишкового кодування з використанням еліптичних кривих, а також їхні переваги у порівнянні з іншими кодами.

\section{Основні поняття}

\section{Алгебро-геометричні коди}

\nocite{*}

\clearpage
\bibliography{coursework}{}
\bibliographystyle{gost71u}


\end{document}
