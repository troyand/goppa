\documentclass[a4paper,12pt,oneside]{article}
\usepackage[left=30mm,right=15mm,top=20mm,bottom=20mm,includefoot]{geometry}
\usepackage[small,compact,center]{titlesec}
\renewcommand{\subsubsection}[1]{\addcontentsline{toc}{subsubsection}{\arabic{section}.\arabic{subsection}.\arabic{
    subsubsection}.~#1}\stepcounter{subsubsection}\textbf{\arabic{section}.\arabic{subsection}.\arabic{subsubsection}.~#1}}
\usepackage{setspace}
\onehalfspacing
\usepackage[utf8]{inputenc}
\usepackage[english,ukrainian]{babel}
\usepackage{verbatim}
\usepackage{indentfirst}
\usepackage{fullpage}
\usepackage{longtable}
\usepackage{amsmath}
\usepackage{cases}
\usepackage{graphicx}
\usepackage{cite}

\title{Курсова робота\\Надлишкове кодування на еліптичних кривих}
\author{Михайло Трояновський}

\begin{document}

\begin{titlepage}
\begin{center}
Національний Університет\\ «Києво-Могилянська Академія»\\
        Кафедра математики\\
        \vskip 3cm
        \includegraphics[width=100mm]{kma}\\
                               \vskip 1cm
{\Large
    Курсова робота на тему:\\
        «Надлишкове кодування на еліптичних кривих»
}
\vskip 1cm
\hskip 0.5\textwidth
\begin{minipage}{0.5\textwidth}
\textbf{Виконав:}\\
        Михайло Михайлович Трояновський, студент ІСПР–1\\
        \\
        \textbf{Науковий керівник:}\\
        Юрій Вікторович Боднарчук
        \end{minipage}
        \vfill
        Київ–2010
        \end{center}
        \end{titlepage}


\tableofcontents
\pagebreak

\section{Вступ}


\section{Постановка задачі}
Мета даної курсової роботи -- описати наявні способи надлишкового кодування з використанням еліптичних кривих, а також їхні переваги у порівнянні з іншими кодами.

\section{Основні поняття}

\section{Алгебро-геометричні коди}

\nocite{*}

\clearpage
\bibliography{coursework}{}
\bibliographystyle{gost71u}


\end{document}
