\documentclass[a4paper,12pt,oneside]{article}
\usepackage[left=30mm,right=15mm,top=20mm,bottom=20mm,includefoot]{geometry}
\usepackage[small,compact,center]{titlesec}
\renewcommand{\subsubsection}[1]{\addcontentsline{toc}{subsubsection}{\arabic{section}.\arabic{subsection}.\arabic{
    subsubsection}.~#1}\stepcounter{subsubsection}\textbf{\arabic{section}.\arabic{subsection}.\arabic{subsubsection}.~#1}}
\usepackage{setspace}
\onehalfspacing
\usepackage[utf8]{inputenc}
\usepackage[english,ukrainian]{babel}
\usepackage{verbatim}
\usepackage{indentfirst}
\usepackage{fullpage}
\usepackage{longtable}
\usepackage{amsmath}
\usepackage{cases}
\usepackage{graphicx}
\usepackage{cite}

\title{Курсова робота\\Надлишкове кодування на еліптичних кривих}
\author{Михайло Трояновський}

\begin{document}

\begin{titlepage}
\begin{center}
Національний Університет\\ «Києво-Могилянська Академія»\\
        Кафедра математики\\
        \vskip 3cm
        \includegraphics[width=100mm]{kma}\\
                               \vskip 1cm
{\Large
    Курсова робота на тему:\\
        «Надлишкове кодування на еліптичних кривих»
}
\vskip 1cm
\hskip 0.5\textwidth
\begin{minipage}{0.5\textwidth}
\textbf{Виконав:}\\
        Михайло Михайлович Трояновський, студент ІСПР–1\\
        \\
        \textbf{Науковий керівник:}\\
        Юрій Вікторович Боднарчук
        \end{minipage}
        \vfill
        Київ–2010
        \end{center}
        \end{titlepage}


\tableofcontents
\pagebreak

\section{Вступ}
Надлишкове кодування відіграє значну роль у боротьбі з помилками, які виникають при передачі даних через зашумлені телекомунікаційні канали. 
Загалом для надійної передачі інформації через ненадійні канали застосовують дві стратегії: виявлення та виправлення помилок. 
У першому випадку до блоку даних додають контрольну інформацію, за допомогою якої одержувач має змогу зрозуміти, 
що під час зв'язку відбулося пошкодження відповідного блоку, тому необхідно зробити запит на повторну передачу. 
Найпростішим прикладом такої контрольної інформації є біт парності, який додається до блоку бінарних даних таким чином, аби загальна кількість одиниць у блоці була парною. 
Це дає можливість одержувачу виявити помилку в одному біті переданого блоку, проте якщо помилок було більше, вони можуть скасувати одна одну; 
втім подібний метод можна досить просто узагальнити для виявлення будь-якої непарної кількості помилок.

\section{Постановка задачі}
Мета даної курсової роботи -- описати наявні способи надлишкового кодування з використанням еліптичних кривих, а також їхні переваги у порівнянні з іншими кодами.

\section{Основні поняття}

\section{Алгебро-геометричні коди}

\nocite{*}

\clearpage
\bibliography{coursework}{}
\bibliographystyle{gost71u}


\end{document}
