\documentclass[a4paper,12pt,oneside]{article}
\usepackage[left=30mm,right=15mm,top=20mm,bottom=20mm,includefoot]{geometry}
\usepackage[small,compact,center]{titlesec}
\renewcommand{\subsubsection}[1]{\addcontentsline{toc}{subsubsection}{\arabic{section}.\arabic{subsection}.\arabic{
    subsubsection}.~#1}\stepcounter{subsubsection}\textbf{\arabic{section}.\arabic{subsection}.\arabic{subsubsection}.~#1}}
\usepackage{setspace}
\onehalfspacing
\usepackage[utf8]{inputenc}
\usepackage[english,ukrainian]{babel}
\usepackage{verbatim}
\usepackage{indentfirst}
\usepackage{fullpage}
\usepackage{longtable}
\usepackage{amsmath}
\usepackage{cases}
\usepackage{graphicx}
\usepackage{cite}

\title{Курсова робота\\Надлишкове кодування на еліптичних кривих}
\author{Михайло Трояновський}

\begin{document}

\maketitle
\tableofcontents
\pagebreak

\section{Вступ}


\section{Постановка задачі}
Мета даної курсової роботи -- описати наявні способи надлишкового кодування з використанням еліптичних кривих, а також їхні переваги у порівнянні з іншими кодами.

\section{Основні поняття}

\section{Алгебро-геометричні коди}

\nocite{*}

\clearpage
\bibliography{coursework}{}
\bibliographystyle{gost71u}


\end{document}
