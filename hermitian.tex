\documentclass[a4paper,12pt,oneside]{article}
%\usepackage[left=20mm,right=15mm,top=20mm,bottom=20mm,includefoot]{geometry}
\usepackage[small,compact,center]{titlesec}
\usepackage{setspace}
\usepackage[T1]{fontenc}
\onehalfspacing
\usepackage[utf8]{inputenc}
\usepackage[english,ukrainian]{babel}
\usepackage{verbatim}
\usepackage{indentfirst}
\usepackage{fullpage}
\usepackage{longtable}
\usepackage{amsmath}
\usepackage{amssymb}
\usepackage{verbatim}
\usepackage{cases}
\usepackage{graphicx}
\usepackage{cite}
\usepackage[14pt]{extsizes}

\title{Надлишкове кодування на алгебраїчних кривих}
\author{Михайло Трояновський}

\begin{document}

\section{Preliminaries}
Нехай $m$ --- степінь простого числа. Тоді над полем $\mathbb{F}_{m^2}$ крива, яка визначається рівнянням $H_m = x^{m+1} + y^m z + y z^m$ ,
називається ермітовою кривою. Ця крива має $m^3+1$ $\mathbb{F}_{m^2}$-раціональних точок, серед них є одна нескінченно віддалена, 
$Q = (0:1:0)$. Рід цієї кривої можна обчислити як $g = \frac{m(m-1)}{2}$.

Нерівність Серра дає оцінку максимальної кількості точок $N$ на кривій роду $g$ над полем $\mathbb{F}_q$:
$$ N \le q+1+g \lfloor 2 \sqrt{q} \rfloor $$
Підставивши параметри ермітової кривої у цю нерівність, отримаємо:
$$ N \le m^2 + 1 + \frac{m(m-1)}{2} \lfloor 2 \sqrt{m^2} \rfloor= m^2 + 1 + m(m-1)m = $$
$$ = m^2 + 1 + m^3 - m^2 = m^3 + 1$$
Звідси зрозуміло, що ермітова крива є максимальною.

Для ермітової кривої базис простору $L(aQ)$ має особливий вигляд та може бути досить тривіально обчислений:
$$ L(aQ) = span \left \{ \frac{x^i y^j}{z^{i+j}}, i m + j (m + 1) \le a, i \le m \right \} $$


\section{Алгоритм}
Нехай $\mathcal{P}$ --- множина усіх $\mathbb{F}_{m^2}$-раціональних точок $H_m$, окрім $Q$, тоді алгеброгеометричний код 
$C = (H_m, \mathcal{P}, aQ)_\Omega$ називатимемо ермітовим кодом з параметрами $(m, a)$.
\subsection{Кодування}
\subsubsection{Попередня підготовка}
Обчислимо перевірочну матрицю $H$
$$ 
H = 
\begin{pmatrix}
 f_1 (P_1) & \cdots & f_1(P_{m^3}) \\
 \vdots & \ddots & \vdots \\
 f_i (P_1) & \cdots & f_i(P_{m^3})
\end{pmatrix}
$$
\subsubsection{Кодування вхідного слова}
\subsection{Декодування}
\subsubsection{Попередня підготовка}
\subsubsection{Декодування вхідного слова}


\section{Аналіз складності}
\section{Приклад}

\end{document}
