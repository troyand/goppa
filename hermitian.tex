\documentclass[a4paper,12pt,oneside]{article}
%\usepackage[left=20mm,right=15mm,top=20mm,bottom=20mm,includefoot]{geometry}
\usepackage[small,compact,center]{titlesec}
\usepackage{setspace}
\usepackage[T1]{fontenc}
\onehalfspacing
\usepackage[utf8]{inputenc}
\usepackage[english,ukrainian]{babel}
\usepackage{verbatim}
\usepackage{indentfirst}
\usepackage{fullpage}
\usepackage{longtable}
\usepackage{amsmath}
\usepackage{amssymb}
\usepackage{verbatim}
\usepackage{cases}
\usepackage{graphicx}
\usepackage{cite}
\usepackage[14pt]{extsizes}
\usepackage{simplemargins}
\setallmargins{22mm}


\title{Надлишкове кодування на алгебраїчних кривих}
\author{Михайло Трояновський}

\begin{document}

\section{Preliminaries}
Нехай $m$ --- степінь простого числа. Тоді над полем $\mathbb{F}_{m^2}$ крива, яка визначається рівнянням $H_m = x^{m+1} + y^m z + y z^m$ ,
називається ермітовою кривою. Ця крива має $m^3+1$ $\mathbb{F}_{m^2}$-раціональних точок, серед них є одна нескінченно віддалена, 
$Q = (0:1:0)$. Рід цієї кривої можна обчислити як $g = \frac{m(m-1)}{2}$.

Нерівність Серра дає оцінку максимальної кількості точок $N$ на кривій роду $g$ над полем $\mathbb{F}_q$:
$$ N \le q+1+g \lfloor 2 \sqrt{q} \rfloor $$
Підставивши параметри ермітової кривої у цю нерівність, отримаємо:
$$ N \le m^2 + 1 + \frac{m(m-1)}{2} \lfloor 2 \sqrt{m^2} \rfloor= m^2 + 1 + m(m-1)m = $$
$$ = m^2 + 1 + m^3 - m^2 = m^3 + 1$$
Звідси зрозуміло, що ермітова крива є максимальною.

Для ермітової кривої базис простору $L(aQ)$ має особливий вигляд та може бути досить тривіально обчислений:
$$ L(aQ) = span \left. \left \{ \frac{x^i y^j}{z^{i+j}} \right | i m + j (m + 1) \le a, i \le m \right \} $$


\section{Алгоритм}
Нехай $\mathcal{P}$ --- множина усіх $\mathbb{F}_{m^2}$-раціональних точок $H_m$, окрім $Q$, тоді алгеброгеометричний код 
$C_{H_{m}, a} = (H_m, \mathcal{P}, aQ)_\Omega$ називатимемо ермітовим кодом з параметрами $(m, a)$. Параметри цього лінійного коду:
$[m^3, m^3 - a + \frac{m(m-1)}{2} - 1, a - m(m-1) + 2]_{\mathbb{F}_{m^2}}$ за умови, що $ a > m(m-1) - 2$.
\subsection{Кодування}
\subsubsection{Попередня підготовка}
Обчислимо базис $L(aQ)$, за теоремою Рімана-Роха кількість його елементів дорівнюватиме: $\ell_{aQ}=a+1-\frac{m(m-1)}{2}$:
$$ basis(L(aQ)) = \left. \left \{ \frac{x^i y^j}{z^{i+j}} \right | i m + j (m + 1) \le a, i \le m \right \} $$

Обчислимо множину раціональних точок $\mathcal{P}$.

Обчислимо перевірочну матрицю $H$:
$$ 
H = 
\begin{pmatrix}
 f_1 (P_1) & \cdots & f_1(P_{m^3}) \\
 \vdots & \ddots & \vdots \\
 f_{\ell_{aQ}} (P_1) & \cdots & f_{\ell_{aQ}}(P_{m^3})
\end{pmatrix}, P_i \in \mathcal{P}, f_j \in basis(L(aQ))
$$

Обчислимо породжуючу матрицю $G$:
$$
G =
\begin{pmatrix}
    G_1 \\
    \vdots \\
    G_k
\end{pmatrix}, G_i \in basis(kernel(H^T))
$$

Зауваження: $(H_m, \mathcal{P}, aQ)_\Omega^\perp = (H_m, \mathcal{P}, (m^3+m^2-m-a-2)Q)_L$, з цих міркувань можна обчислювати матриці $H$ і $G$ як матриці 
$G'$ та $H'$ відповідного дуального коду.

\subsubsection{Кодування вхідного слова}
Для вхідного слова $v, v\in \mathbb{F}_{m^2}^k$ обчислимо кодове слово $c, c \in \mathbb{F}_{m^2}^n$:
$$c = v \cdot G$$
 
\subsection{Декодування алгоритмом Скоробогатова-Вледуца}
\subsubsection{Попередня підготовка}
Кількість помилок, які може виправити алгоритм: $t = \left \lfloor \frac{a-\frac{3m(m-1)}{2} + 1}{2} \right \rfloor$

Обчислимо базис $\{g_j\}$ простору $L( (t+g)Q )$ та обчислимо базис $\{h_k\}$ простору $L( (a-t-g)Q)$.

Обчислимо матрицю синдромів $S$:
$$
S = 
\begin{pmatrix}
    g_1 \\
    \vdots \\
    g_{\ell( (t+g)Q )}
\end{pmatrix}
\cdot
\begin{pmatrix}
    h_1 & \cdots & h_{\ell( (a-t-g)Q)}
\end{pmatrix}
$$

\subsubsection{Декодування вхідного слова}
Для вхідного слова $v, v=c+e, v \in \mathbb{F}_{m^2}^n$ обчислимо матрицю синдромів слова:
$$
S_v = 
\begin{pmatrix}
    g_1 f_1 \cdot v & g_2 f_1 \cdot v & \dots \\
    g_1 f_2 \cdot v & g_2 f_2 \cdot v & \dots \\
    \vdots & \vdots & \ddots
\end{pmatrix}
$$
Тут $\phi \cdot v = (\phi(P_1), \dots , \phi(P_{m^3})) \cdot v$

Якщо отримана матриця складається із нулів, тоді вектор $e$ є нульовим, а слово $v$ --- кодовим.

Знайдемо ядро $S_v$, якщо базис відповідного простору складається не з одного вектора, алгоритм не може декодувати слово $v$.
Нехай $(s_1, \dots, s_{\ell( (t+g)Q )})$ --- вектор відповідного базису. Обчислимо скалярний добуток:
$$\theta = (s_1, \dots, s_{\ell( (t+g)Q )}) \cdot (g_1, \dots, g_{\ell( (t+g)Q )})$$

Знайдемо координати $i_e$, у яких відбулися помилки: перебором $P_{i_e} \in \mathcal{P}$ знаходимо такі точки, для яких $\theta(P_{i_e})=0$.

Вирішуємо систему рівнянь:
$$
\begin{pmatrix}
    f_1 ( P_{i_{e_1}} ) & \dots & f_1 ( P_{i_{e_t}} ) \\
    \vdots & \ddots & \vdots \\
    f_{\ell(aQ)} ( P_{i_{e_1}} ) & \dots & f_{\ell(aQ)} ( P_{i_{e_t}} )
\end{pmatrix} \cdot 
\begin{pmatrix}
    e_1 \\
    \vdots \\
    e_t
\end{pmatrix} = 
\begin{pmatrix}
    f_1 \cdot v \\
    \vdots \\
    f_{\ell(aQ)} \cdot v
\end{pmatrix}, f_i \in L(aQ)
$$

Отримавши значення $e_{i_e}$, відновимо вхідний вектор: $c = v - e$.

Декодоване слово $w$ отримаємо із співвідношення $w \cdot G = c$.



\section{Приклад}
Для прикладу візьмемо ермітовий код з параметрами $(2,6)$. Відповідна крива $H_2$ задається рівнянням $x^3 + y^2 z + y z^2$ над полем $\mathbb{F}_4$. 
Окрім нескінченно віддаленої точки $Q = (0 : 1 : 0)$ у неї 8 раціональних точок:

$\mathcal{P} = \{ (0 : 0 : 1), (0 : 1 : 1), (1 : w : 1), (1 : w + 1 : 1), (w : w : 1), \\
 (w : w + 1 : 1), (w + 1 : w : 1), (w + 1 : w + 1 : 1)\}$

Ця крива має рід 1 (більше того, це еліптична крива), отже параметри лінійного коду $[8, 2, 6]_{\mathbb{F}_4}$, кількість помилок, які може виправити 
алгоритм Скоробогатова-Вледуца $t=2$.

$$L(6Q) = span \left \{ 1, y/z, y^2/z^2, x/z, xy/z^2, x^2/z^2 \right \}$$

Породжуюча матриця може бути обчислена як перевірочна відповідного дуального коду:
$$
G = 
\begin{pmatrix}
    1 &  1 &  1 &  1 &  1 &  1 &  1 &  1 &  \\
    0 &  0 &  1 &  1 &  w &  w &  w + 1 &  w + 1 &  \\
\end{pmatrix}
$$
$$
H = 
\begin{pmatrix}
    1 &  0 &  0 &  0 &  0 &  w + 1 &  0 &  w &  \\
    0 &  1 &  0 &  0 &  0 &  w + 1 &  0 &  w &  \\
    0 &  0 &  1 &  0 &  0 &  w &  0 &  w + 1 &  \\
    0 &  0 &  0 &  1 &  0 &  w &  0 &  w + 1 &  \\
    0 &  0 &  0 &  0 &  1 &  1 &  0 &  0 &  \\
    0 &  0 &  0 &  0 &  0 &  0 &  1 &  1 &  \\
\end{pmatrix}
$$

Нехай вхідне повідомлення $(0,1)$, тоді кодове слово:
$$c=(0, 0, 1, 1, w, w, w + 1, w + 1)$$

Нехай при передачі відбулись помилки у другій та восьмій позиціях і приймальна сторона отримала слово:
$$v=(0, 1, 1, 1, w, w, w + 1, 0)$$

Матриця синдромів для коду має вигляд:
$$
S =
\begin{pmatrix}
    1 &  y/z &  x/z &  \\
    y/z &  y^2/z^2 &  xy/z^2 &  \\
    x/z &  xy/z^2 &  x^2/z^2 &  \\
\end{pmatrix}
$$

Для отриманого вектора $v$ вона набуде вигляду:
$$
S_v =
\begin{pmatrix}
    w & w +1 & w \\
    w + 1 & 0 & 1 \\
    w & 1 & 1 \\
\end{pmatrix}
$$

Локатор помилок:
$$
\theta = \frac{(( w + 1)*x + y + z)}{z}
$$

\section{Аналіз складності}
Через те, що параметр $m$ накладає певні обмеження на параметр $a$ ермітового коду, складність алгоримтму 
можна виразити лише через параметр $m$.

При побудові коду, а також при декодуванні необхідно обчислювати базиси трьох просторів Рімана-Роха: $L(aQ), L( (t+g)Q ), L( (a-t-g)Q)$. 
Для того, аби в теоремі Рімана-Роха досягалась рівність, необхідно накласти такі обмеження на параметри (для спрощення обчислень у виразі параметру $t$ 
буде прибрано округлення вниз, це може вплинути на точність оцінки при малих $m$):
$$
\begin{cases}
    a > 2g - 2 \\
    t+g > 2g - 2 \\
    a - t -g > 2g -2
\end{cases},
\begin{cases}
    t > g - 2 \\
    a > 3g + t - 2
\end{cases},
\begin{cases}
    \frac{a-3g+1}{2} > g - 2 \\
    a > 3g + \frac{a-3g+1}{2} - 2
\end{cases},
$$
$$
\begin{cases}
    a-3g+1 > 2g - 4 \\
    2a > 6g + a-3g+1 - 4
\end{cases},
\begin{cases}
    a > 5g - 5 \\
    a > 3g - 3
\end{cases},
a > 5 \frac{m(m-1)}{2} - 5
$$

З іншого боку має виконуватись нерівність $0 < k < n$, тобто $0 < m^3 - a - \frac{m(m-1)}{2} - 1 < m^3$, яка накладає обмеження:
$$
\begin{cases}
    a > \frac{m(m-1)}{2} - 1 \\
    a < m^3 + \frac{m(m-1)}{2} - 1
\end{cases}
$$

Підсумовуючи наведені нерівності, можна сказати, що $a$ має лежати у такому проміжку:
$$5 \frac{m(m-1)}{2} - 5 < a < m^3 + \frac{m(m-1)}{2} - 1,$$
тобто асимптотично $a$ має зростати швидше, ніж $m^2$, але повільніше, ніж $m^3$, тому далі можна вважати, що $a=O(m^3)$ і $a=\Omega(m^2)$.

Більшість етапів алгоритму використовує операції із матрицями над скінченним полем, тому відповідні оцінки складності будуть прив'язані 
до складності операцій в полі. При використанні представлення елементів $\mathbb{F}_{m^2}$ у вигляді лишків незвідного полінома операції 
множення та ділення потребуватимуть $O(\ln^2 m^2)=O(\ln^2 m)$ операцій, піднесення елемента поля до додатнього степеню $N$ потребуватиме 
$O(\ln N \ln^2 m)$ операцій. Додавання двох елементів поля потребуватиме $O(\ln m)$ операцій. Використання представлення елементів поля у 
вигляді степенів породжуючого елемента зведе операції множення та ділення до додавання та віднімання показників степенів за модулем $m^2-1$, 
складність такої операції дорівнюватиме $O(\ln(m))$. Проте така форма представлення незручна для додавання та віднімання елементів, тому 
необхідно попередньо створити таблицю відповідностей двох форм запису, це потребуватиме $O(m^2 \ln^2 m)$ операцій та $O(m^2)$ пам'яті.

Для ермітової кривої процедуру знаходження базису $L(aQ)$ можна зводиться до знаходження пар $(i, j)$, для яких виконується співвідношення 
$i m + j (m + 1) \le n$ у циклах, пробігаючи $0 \le j \le \left \lfloor \frac{a}{m+1} \right \rfloor$ та $0 \le i \le m$. Тобто усього $O(a)$ операцій 
вартістю порядку $O(\ln a)$ кожна, загалом $O(a \ln a)$.

Для побудови перевірочної матриці $H$ необхідно знайти базис $L(aQ)$, це займе $O(a \ln a) = O(m^3 \ln m)$ операцій. 
Перебірна реалізація знаходження раціональних точок (підстановка усіх $x, y \in \mathbb{F}_{m^2}$ у рівняння кривої, найдорожча арифметична операція --- 
піднесення елемента поля до $m$-го степеня) займе $O(m^2 m^2 m \ln^2 m) = O(m^5 \ln^2 m)$. Отже, у матриці $H$ $\ell_{aQ}=a+1-\frac{m(m-1)}{2}$ рядків та 
$m^3$ стовпчиків, тому загальну кількість її елементів можна оцінити як $O(m^3 m^3) = O(m^6)$. Для обчислення кожного з її елементів найдорожчою 
арифметичною операцією буде піднесення до степеню (в найгіршому випадку $\frac{a}{m+1}=O(m^2)$), складність операції $O(\ln m^2 \ln^2 m) = O(\ln^3 m)$. 
Тому загальна складність обчислення всіх елементів $H$ буде $O(m^6 \ln^3 m)$, як бачимо це і буде найдорожчою операцією у побудові $H$. Із 
властивості самодуальності ермітового коду вважатимемо, що складність побудови матриці $G$ не перевищуватиме складності побудови $H$.

Кодування полягає у множенні вектора довжини $k$ на матрицю $k \times n$: для цього необхідно обчислити $k n$ результатів множення елементів поля: 
складність $k n \ln^2 m = O(m^3 m^3 \ln^2 m) = O(m^6 \ln^2 m)$.

При декодуванні найдорожчою операцією буде обчислення матриці $S_v$, кількість її елементів можна оцінити як $O(m^6)$, обчислення кожного її елементу полягає 
в обчисленні скалярного добутку двох векторів довжиною $m^3$. Отже, загальна складність знаходження $S_v$ дорівнюватиме $O(m^9 \ln^2 m)$.

Вартість розв'язку систем лінійних рівнянь, яка виникає при декодуванні має аналогічну складність (матриця розміром $m^3 \times m^3$): 
метод Гаусса виглядатиме так: $i$-тий рядок ділимо на елемент $a_{i,i}$, для кожного іншого $k$-го рядка віднімаємо даний, поділений на $a_{k,i}$:
це буде три вкладені цикли по $m^3$ кроків у кожному із вартістю внутрішньої операції $O(\ln^2 m)$. Загальна складність методу: $O(m^9 \ln^2 m)$.

Наведена складність алгоритму Скоробогатова-Вледуца збігається із оцінкою, яка дається у літературі: $O(n^3)$ операцій в скінченному полі. 
Якщо врахувати, що для ермітового коду $n=m^3$, а вартість множення $O(\ln^2 m)$, ми отримуємо оцінку $O(m^9 \ln^2 m)$.

\end{document}
